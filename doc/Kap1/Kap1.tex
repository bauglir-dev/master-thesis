\chapter{Introducci\'{o}n}


%Corrupcion desde la econimia

Utilizaci\'{o}n del sistema financiero global para ocultar el dinero obtenido ilegalmente. \cite{granados2022geometry}\\

La corrupci\'{o}n es un fen\'{o}meno que afecta sociedades de todo tipo, en todo lugar y de toda \'{e}poca. existe la percepci\'{o}n de que ciertas sociedades son m\'{a}s proclives a ese fen\'{o}meno, sin embargo, estudios como el de Ariely  \cite{ariely2017corruption} y el de Anderson \cite{anderson2003corruption} muestran que la relaci\'{o}n entre percepci\'{o}n y corrupci\'{o}n es m\'{a}s complicada. Por un lado, puede verse distorsionada cuando se considera la relativa equidad o iniquidad del sistema al que pertenece el sujeto. Por otro lado, la presencia de corrupci\'{o}n no necesariamente disminuye el apoyo de los ciudadanos a sus instituciones.\\

Pueden identificarse tres pilares en la normalizaci\'{o}n de la corrupci\'{o}n: (1) {\it institucionalizaci\'{o}n}, (2) {\it racionalizaci\'{o}n} y (3) {\it socializaci\'{o}n} \cite{ashforth2003normalization}. Durante la institucionalizaci\'{o}n, las pr\'{a}cticas corruptas son promulgadas como rutinas a menudo en forma consciente para aquellos que la exhiben. En la fase de racionalizaci\'{o}n, individuos que cometen actos corruptos utilizan construcciones sociales para legitimar el acto ante sus propios ojos. En el proceso de socializaci\'{o}n, las practica corruptas son difundidas y aceptadas por nuevos integrantes.\\

%Corrupcion desde la teoria de la complejidad
Algunos autores han sugerido la adopci\'{o}n de nuevos paradigmas para el estudio de la corrupci\'{o}n. En particular, se ha observado el potencial de utilizar la teor\'{\i}a de la complejidad \cite{nicolas2021corruptomics}.  La idea principal en el estudio de sistemas complejos es que las interacciones importan m\'{a}s que la naturaleza de las partes. Tambi\'{e}n pregona que el comportamiento de un sistema complejo no puede predecirse a partir del estudio de las partes desconectadas. Cuando los comportamientos corruptos se manifiestan, lo hacen a trav\'{e}s de ocultas y complicadas interrelaciones entre m\'{u}ltiples actividades incrustadas en estructuras sociales, econ\'{o}micas, pol\'{\i}ticas y t\'{e}cnicas. En un sistema complejo, resulta de mayor inter\'{e}s el comportamiento colectivo y las caracter\'{\i}sticas de los individuos conectados m\'{a}s que las caracter\'{\i}sticas particulares de los elementos de la red. En otras palabras, la corrupci\'{o}n es un acto colectivo que ocurre en el contexto de organizaciones \cite{ashforth2003normalization} y no necesariamente es llevada a cabo por individuos solitarios.\\

Pueden definirse y adoptarse marcos de trabajo para entender los fen\'{o}menos de corrupci\'{o}n desde el \'{a}mbito de la complejidad. Algunos autores ya han propuesto la teor\'{\i}a de redes como fundamento del an\'{a}lisis estos fen\'{o}menos. En el trabajo de Luna-Pla \cite{luna2020corruption} se propone un marco de trabajo para los estudios modernos de corrupci\'{o}n basado en teor\'{\i}a de redes: definici\'{o}n, medici\'{o}n, predicci\'{o}n y control. Este marco se aplica a un caso de corrupci\'{o}n en M\'{e}xico ocurrido entre el 2010 y el 2016. Una de las conclusiones m\'{a}s relevantes de este trabajo hace \'{e}nfasis en la selecci\'{o}n adecuada de m\'{e}tricas de redes que permitan identificar anomal\'{\i}as con mayor precisi\'{o}n y gu\'{\i}en la construcci\'{o}n de indicadores de corrupci\'{o}n efectivos.\\






\iffalse
En la introducci\'{o}n, el autor presenta y se\~{n}ala la importancia, el origen (los antecedentes te\'{o}ricos y pr\'{a}cticos), los objetivos, los alcances, las limitaciones, la metodolog\'{\i}a empleada, el significado que el estudio tiene en el avance del campo respectivo y su aplicaci\'{o}n en el \'{a}rea investigada. No debe confundirse con el resumen y se recomienda que la introducci\'{o}n tenga una extensi\'{o}n de m\'{\i}nimo 2 p\'{a}ginas y m\'{a}ximo de 4 p\'{a}ginas.\\

La presente plantilla maneja una familia de fuentes utilizada generalmente en LaTeX, conocida como Computer Modern, espec\'{\i}ficamente LMRomanM para el texto de los p\'{a}rrafos y CMU Sans Serif para los t\'{\i}tulos y subt\'{\i}tulos. Sin embargo, es posible sugerir otras fuentes tales como Garomond, Calibri, Cambria, Arial o Times New Roman, que por claridad y forma, son adecuadas para la edici\'{o}n de textos acad\'{e}micos.\\

La presente plantilla tiene en cuenta aspectos importantes de la Norma T\'{e}cnica Colombiana - NTC 1486, con el fin que sea usada para la presentaci\'{o}n final de las tesis de maestr\'{\i}a y doctorado y especializaciones y especialidades en el \'{a}rea de la salud, desarrolladas en la Universidad Nacional de Colombia.\\

Las m\'{a}rgenes, numeraci\'{o}n, tama\~{n}o de las fuentes y dem\'{a}s aspectos de formato, deben ser conservada de acuerdo con esta plantilla, la cual esta dise\~{n}ada para imprimir por lado y lado en hojas tama\~{n}o carta. Se sugiere que los encabezados cambien seg\'{u}n la secci\'{o}n del documento (para lo cual esta plantilla esta construida por secciones).\\

Si se requiere ampliar la informaci\'{o}n sobre normas adicionales para la escritura se puede consultar la norma NTC 1486 en la Base de datos del ICONTEC (Normas T\'{e}cnicas Colombianas) disponible en el portal del SINAB de la Universidad Nacional de Colombia\footnote{ver: www.sinab.unal.edu.co}, en la secci\'{o}n "Recursos bibliogr\'{a}ficos" opci\'{o}n "Bases de datos".  Este portal tambi\'{e}n brinda la posibilidad de acceder a un instructivo para la utilizaci\'{o}n de Microsoft Word y Acrobat Professional, el cual est\'{a} disponible en la secci\'{o}n "Servicios", opci\'{o}n "Tr\'{a}mites" y enlace "Entrega de tesis".\\

La redacci\'{o}n debe ser impersonal y gen\'{e}rica. La numeraci\'{o}n de las hojas sugiere que las p\'{a}ginas preliminares se realicen en n\'{u}meros romanos en may\'{u}scula y las dem\'{a}s en n\'{u}meros ar\'{a}bigos, en forma consecutiva a partir de la introducci\'{o}n que comenzar\'{a} con el n\'{u}mero 1. La cubierta y la portada no se numeran pero si se cuentan como p\'{a}ginas.\\

Para trabajos muy extensos se recomienda publicar m\'{a}s de un volumen. Se debe tener en cuenta que algunas facultades tienen reglamentada la extensi\'{o}n m\'{a}xima de las tesis  o trabajo de investigaci\'{o}n; en caso que no sea as\'{\i}, se sugiere que el documento no supere 120 p\'{a}ginas.\\

No se debe utilizar numeraci\'{o}n compuesta como 13A, 14B \'{o} 17 bis, entre otros, que indican superposici\'{o}n de texto en el documento. Para resaltar, puede usarse letra cursiva o negrilla. Los t\'{e}rminos de otras lenguas que aparezcan dentro del texto se escriben en cursiva.\\

\fi