\chapter{Introducción}


%Corrupcion desde la econimia

Utilización del sistema financiero global para ocultar el dinero obtenido ilegalmente. \cite{granados2022geometry}\\

La corrupción es un fenómeno que afecta sociedades de todo tipo, en todo lugar y de toda época. existe la percepción de que ciertas sociedades son más proclives a ese fenómeno, sin embargo, estudios como el de Ariely  \cite{ariely2017corruption} y el de Anderson \cite{anderson2003corruption} muestran que la relación entre percepción y corrupción es más complicada. Por un lado, puede verse distorsionada cuando se considera la relativa equidad o iniquidad del sistema al que pertenece el sujeto. Por otro lado, la presencia de corrupción no necesariamente disminuye el apoyo de los ciudadanos a sus instituciones.\\

Pueden identificarse tres pilares en la normalización de la corrupción: (1) \textit{institucionalización}, (2) \textit{racionalización} y (3) \textit{socialización} \cite{ashforth2003normalization}. Durante la institucionalización, las prácticas corruptas son promulgadas como rutinas a menudo en forma consciente para aquellos que la exhiben. En la fase de racionalización, individuos que cometen actos corruptos utilizan construcciones sociales para legitimar el acto ante sus propios ojos. En el proceso de socialización, las practica corruptas son difundidas y aceptadas por nuevos integrantes.\\

%Corrupcion desde la teoria de la complejidad
Algunos autores han sugerido la adopción de nuevos paradigmas para el estudio de la corrupción. En particular, se ha observado el potencial de utilizar la teoría de la complejidad \cite{nicolas2021corruptomics}.  La idea principal en el estudio de sistemas complejos es que las interacciones importan más que la naturaleza de las partes. También pregona que el comportamiento de un sistema complejo no puede predecirse a partir del estudio de las partes desconectadas. Cuando los comportamientos corruptos se manifiestan, lo hacen a través de ocultas y complicadas interrelaciones entre múltiples actividades incrustadas en estructuras sociales, económicas, políticas y técnicas. En un sistema complejo, resulta de mayor interés el comportamiento colectivo y las características de los individuos conectados más que las características particulares de los elementos de la red. En otras palabras, la corrupción es un acto colectivo que ocurre en el contexto de organizaciones \cite{ashforth2003normalization} y no necesariamente es llevada a cabo por individuos solitarios.\\

Pueden definirse y adoptarse marcos de trabajo para entender los fenómenos de corrupción desde el ámbito de la complejidad. Algunos autores ya han propuesto la teoría de redes como fundamento del análisis estos fenómenos. En el trabajo de Luna-Pla \cite{luna2020corruption} se propone un marco de trabajo para los estudios modernos de corrupción basado en teoría de redes: definición, medición, predicción y control. Este marco se aplica a un caso de corrupción en México ocurrido entre el 2010 y el 2016. Una de las conclusiones más relevantes de este trabajo hace énfasis en la selección adecuada de métricas de redes que permitan identificar anomalías con mayor precisión y guíen la construcción de indicadores de corrupción efectivos.\\






\iffalse
En la introducción, el autor presenta y se\~{n}ala la importancia, el origen (los antecedentes teóricos y prácticos), los objetivos, los alcances, las limitaciones, la metodología empleada, el significado que el estudio tiene en el avance del campo respectivo y su aplicación en el área investigada. No debe confundirse con el resumen y se recomienda que la introducción tenga una extensión de mínimo 2 páginas y máximo de 4 páginas.\\

La presente plantilla maneja una familia de fuentes utilizada generalmente en LaTeX, conocida como Computer Modern, específicamente LMRomanM para el texto de los párrafos y CMU Sans Serif para los títulos y subtítulos. Sin embargo, es posible sugerir otras fuentes tales como Garomond, Calibri, Cambria, Arial o Times New Roman, que por claridad y forma, son adecuadas para la edición de textos académicos.\\

La presente plantilla tiene en cuenta aspectos importantes de la Norma Técnica Colombiana - NTC 1486, con el fin que sea usada para la presentación final de las tesis de maestría y doctorado y especializaciones y especialidades en el área de la salud, desarrolladas en la Universidad Nacional de Colombia.\\

Las márgenes, numeración, tama\~{n}o de las fuentes y demás aspectos de formato, deben ser conservada de acuerdo con esta plantilla, la cual esta dise\~{n}ada para imprimir por lado y lado en hojas tama\~{n}o carta. Se sugiere que los encabezados cambien según la sección del documento (para lo cual esta plantilla esta construida por secciones).\\

Si se requiere ampliar la información sobre normas adicionales para la escritura se puede consultar la norma NTC 1486 en la Base de datos del ICONTEC (Normas Técnicas Colombianas) disponible en el portal del SINAB de la Universidad Nacional de Colombia\footnote{ver: www.sinab.unal.edu.co}, en la sección "Recursos bibliográficos" opción "Bases de datos".  Este portal también brinda la posibilidad de acceder a un instructivo para la utilización de Microsoft Word y Acrobat Professional, el cual está disponible en la sección "Servicios", opción "Trámites" y enlace "Entrega de tesis".\\

La redacción debe ser impersonal y genérica. La numeración de las hojas sugiere que las páginas preliminares se realicen en números romanos en mayúscula y las demás en números arábigos, en forma consecutiva a partir de la introducción que comenzará con el número 1. La cubierta y la portada no se numeran pero si se cuentan como páginas.\\

Para trabajos muy extensos se recomienda publicar más de un volumen. Se debe tener en cuenta que algunas facultades tienen reglamentada la extensión máxima de las tesis  o trabajo de investigación; en caso que no sea así, se sugiere que el documento no supere 120 páginas.\\

No se debe utilizar numeración compuesta como 13A, 14B ó 17 bis, entre otros, que indican superposición de texto en el documento. Para resaltar, puede usarse letra cursiva o negrilla. Los términos de otras lenguas que aparezcan dentro del texto se escriben en cursiva.\\

\fi